% arara: pdflatex: { shell: true }
% arara: pdflatex: { shell: true }
%-------------------------------------------------------------------------------
%	PACKAGES AND OTHER DOCUMENT CONFIGURATIONS
%-------------------------------------------------------------------------------

\documentclass[paper=letter, fontsize=12pt]{article}

\usepackage[english]{babel} % English language/hyphenation
\usepackage{amsmath,amsfonts,amsthm} % Math packages
\usepackage[section]{placeins}  % prevent figures / eqns / tables 
                                % from slipping out of section.
\usepackage{sectsty} % Allows customizing section commands
\allsectionsfont{\centering \normalfont\scshape} % Make all sections centered, 
                                                 % the default font and small 
                                                 % caps 

\usepackage{fancyhdr} % Custom headers and footers
\usepackage{hyperref}

% algorithms
\usepackage{algorithm}
\usepackage[compatible]{algpseudocode}

\usepackage{ragged2e} % Text alignment
\usepackage[margin=1in]{geometry} % 1 inch margins
\usepackage{tikz} % Not always necessary, but allows very customizable 
                  % figures / object placement
\usepackage{enumitem} % get more out of enumerate

% Header and Footers
\pagestyle{fancyplain} % Makes all pages in the document conform to the custom 
                       % headers and footers
\fancyhead{} % No page header
\fancyfoot[L]{} % Empty left footer
\fancyfoot[C]{} % Empty center footer
\fancyfoot[R]{\thepage} % Page numbering for right footer
\renewcommand{\headrulewidth}{0pt} % Remove header underlines
\renewcommand{\footrulewidth}{0pt} % Remove footer underlines
\setlength{\headheight}{13.6pt} % Customize the height of the header

% Number all figs, eqns, and tables within the section
\numberwithin{equation}{section} % Number equations within sections 
\numberwithin{figure}{section} % Number figures within sections 
\numberwithin{table}{section} % Number tables within sections 

\setlength\parindent{4pt} % 4pt indentation for paragraphs
\setlength{\parskip}{\baselineskip} % adds some spacing in between paragraphs

\newcommand{\horrule}[1]{\rule{\linewidth}{#1}} % Create horizontal rule command
                                                % with 1 argument of height
\newcommand{\fancyline}{\\ \horrule{0.5pt} \vspace{0.1cm}} % fancy line to put
                                                           % under questions

%  Some commands for algorithm environments                                            
\renewcommand{\algorithmicrequire}{\textbf{Input:}}
\renewcommand{\algorithmicensure}{\textbf{Output:}}
\renewcommand{\algorithmicforall}{\textbf{for each}}
\newcommand{\algorithmiccontinue}{\textbf{continue}}
\algloopdefx{RETURN}[1][]{\textbf{return} #1}

% useful math shortcuts
\newcommand{\lagr}[1]{\mathcal{L}\left( #1 \right)}
\newcommand{\expval}[1]{E\left[#1\right]}
\newcommand{\var}[1]{\text{var}\left(#1\right)}
\renewcommand{\det}[1]{\text{det}\left(#1\right)}
\newcommand{\diag}[1]{\text{diag}\left[#1\right]}

% Code Cells
  \usepackage{minted}
  \usepackage{mdframed}

% Colors for minted package
\definecolor{codecellbg}{rgb}{0.950,0.950,0.950}
\definecolor{riceblue}{rgb}{0.000,0.141,0.416}
\definecolor{ricegray}{rgb}{0.369,0.376,0.384}

\mdfdefinestyle{inputcell}{%
 hidealllines=true,leftline=true,
 skipabove=12pt,skipbelow=0pt,
 innertopmargin=0.4em,%
 innerbottommargin=0.4em,%
 innerrightmargin=0.5em,%
 rightmargin=0.5em,%
 innerleftmargin=0.5em,%
 leftmargin=0.5em,%
 linecolor=riceblue,linewidth=3pt,%
 backgroundcolor=codecellbg,%
}

% sets up the minted styles, for now hardcodes python.
\setminted{mathescape,
           numbersep=5pt,
           autogobble=true,
           breaklines,
           framesep=2mm}

\BeforeBeginEnvironment{minted}{\begin{mdframed}[style=inputcell]}
\AfterEndEnvironment{minted}{\end{mdframed}}

%-------------------------------------------------------------------------------
%	TITLE SECTION
%-------------------------------------------------------------------------------

\title{	
\normalfont \normalsize 
\textsc{rice university} \\ [25pt]
\horrule{0.5pt} \\[0.4cm] % Thin top horizontal rule
\huge Solitaire Design Documentation \\ % The assignment title
\horrule{2pt} \\[0.5cm] % Thick bottom horizontal rule
}

\author{Gabriel Vacaliuc}

\begin{document}

\maketitle

%-------------------------------------------------------------------------------
%   CONTENT
%-------------------------------------------------------------------------------

\section{how to play}

\subsection{compiling and running}

The code was written for the Java 10 JVM, so you'll need a Java 10 JRE to run 
the provided JAR and a Java 10 JDK to compile the code.  I've packaged the 
game with \href{https://gradle.org/}{gradle}, so you can build test and run the
game like so:

\begin{minted}{bash}
$ ./gradlew build

$ ./gradlew test

$ ./gradlew play
\end{minted}

\subsection{playing}

Rules for Solitaire can be
\href{https://www.bicyclecards.com/how-to-play/solitaire/}{found online}.
Note that the deck of cards is the ``Talon'', the area with the 7 stacks of
cards is the ``Tableau'', and the areas where you try to place the cards are
``Foundations''.

There consist of 6 valid actions in our game:
\begin{itemize}
  \item Advance the Talon: Press enter at the command prompt.
  \item Restart the game: Enter \texttt{restart} or \texttt{r}.
  \item Move a card from the Talon to the Tableau: Enter \texttt{t -> \#},
    where \texttt{\#} is the number of the tableau pile to move to. Note that
    you can also write \texttt{talon} rather than abbreviating.
  \item Move a card from a pile on the Tableau to another pile: Enter
    \texttt{\# -> \#}, where the two \texttt{\#}'s represent the numbers of the
    piles to move to/from.
  \item Move a card from the Talon to the Foundations: Enter \texttt{t -> f}.
    Note that you can also write \texttt{foundation} rather than abbreviating.
  \item Move a card from the Tableau to the Foundations: Enter \texttt{\# -> f},
    where \texttt{\#} is the number of the Tableau pile to move the card from.
\end{itemize}

Note that in the full game of Solitaire you can also move cards from the
Foundations to the Tableau.  This was not implemented.

\section{design choices}

I designed the game using a Model-View-Controller architecture.  The Model
consists of the card, deck, tableau, and foundation implementations as well as
their interactions.  The View is responsible for parsing the player's input as
well as displaying the board.  The view contains an adapter interface that is
able to communicate with the model.  This adapter is made concrete in the
controller, so that we can close over our reference of the model.

I chose to implement the game functionally - that is, the data structures in
the game do not mutate; a pile of cards would return a new pile of cards after
some were picked up from the pile.  This allows for simpler testing and
implementation, as you don't have to worry about the state of the game changing
after method calls.  I also made extensive use of \texttt{java.util.Optional},
which reduces the risk of null pointer exceptions.  Rather than returning null
or throwing an exception upon an error, e.g. a player moves a red card onto a
red card, I'd return an empty \texttt{Optional} to indicate that an invalid
move was played.

\section{rationale behind tooling}

I used Java 10 because I'm comfortable in Java, it's easy to refactor, and I
enjoy Java development, for the most part.  I don't have any real reasons for
using Java 10 over Java 8 or 9.  

I used the standard libraries in Java as well as Google Guava.  I got used to
using it this past summer at Two Sigma and it's quite nice to have.  

I implemented tests using JUnit because I have experience using it, and it's a
nice interface.  Although someone \textit{just} turned my attention to
\href{https://github.com/ncredinburgh/QuickTheories}{QuickTheories}, and it
looks like a very impressive testing framework...

I used gradle for my build system because I have experience using it, it's
generally easier to setup than \href{https://bazel.build/}{Bazel}, and I don't
need Bazel's blazing fast builds since this is a tiny project.  I've also been
using gradle for all my Java work this year, so it was a pretty easy decision.

\section{references}

\begin{itemize}
  \item Google Guava: \href{https://github.com/google/guava}{https://github.com/google/guava}
  \item JUnit: \href{https://github.com/junit-team/junit4}{https://github.com/junit-team/junit4}
\end{itemize}

\end{document}
